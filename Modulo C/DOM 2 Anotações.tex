EVENTOS DOM: 

evento é tudo que pode acontecer com tal tag, vou pegar como exemplo uma <div>, tudo que pode acontecer com esta div.

tipo mause perto da div ou movendo dentro da div.

eventos com mause {
    mauseenter [Acontece quando o mause entra dentro da div]
    mausemove [Acontece quando o mause se move dentro da div]
    mausedown [Acontece quando se clicar e segurar o botão do mause dentro da div]
    mauseup [Acontece quando solta o botão do mause]
    click [Acontece quando aperta e solta rapidamente o botão do mause]
    mauseout [Acontece quando o mause sai da div]
}

FUNÇÃO:
É um conjunto de codigos que só vai acontecer quando um evento acontecer. 

fuction eventoClickDivInteraja(param *opcional*){

}

Para declarar um evento de click no HTML:

exemplo: 
<h1 id="area" onclick="clicarArea()" onmouseenter="entrouArea()" onmouseout="saiuArea()">interaja....</h1>

usamos o comando começando com ON... e o nome do evento, onclick, onmouseenter e etc...

quando uma linha tem muitos eventos assim como exemplo acima usamos no js:
ex: 
var elementArea = document.getElementById('area')
elementArea.addEventListener('click', clicarArea)
 elementArea.addEventListener('mouseenter', entrouArea)
 elementArea.addEventListener('mouseout', saiuArea)

function clicarArea(){
elementArea.innerHTML = "CLICOU!"
elementArea.style.background= "green"
}
function entrouArea() {
elementArea.innerHTML = "hmm..."
}
function saiuArea(){
elementArea.innerHTML= "interaja...."
elementArea.style.background= "red"
}

observe: addEventListener ele precisa esta ligado com a variavel que pega o elemento HTML, 
e ele esta totalmente ligado as functions declaradas, então o uso dele não retira o uso da function,
mas retira o uso totalmente do evento no html como exemplo acima com "onclick"

outro detalhe, no comando addEventListener{
    elementArea.addEventListener('click', clicarArea)
    
    a parte 'click' é o nome do evento, a virgula é para separar do nome da FUNÇÃO.
}
