O que é dom?
É um Document object Model

É um conjunto de objetos dentro do meu navegador que vai dar acesso aos componentes internos do meu website, por isto não funciona no nodejs, somente quando estiver processando js na web

Arvore dom.
Raiz do javaScrip se chama ... window
window{
    location //localização do meu site, url, a pagina atual[]

    document //documento atual [
        HTML(
                HEAD {
                    meta, TITLE, todas tags que estao dentro da tag head
                }
                BODY{
                    h1, p, div...,  todas tags que estao dentro da tag BODY
                }
    )]

    history //que vai guardando toda movimentação dentro do site []
}

Existe varias formas de acessar cada elemento existente na Arvore DOM, e este modos são:
1- por Marca (por tag name - getElementsByTagName()[0])
2- por ID (por id - getElementById())
3- por NOME (por nome - getElementsByName()[1])
4- por CLASSE (por CLASSE - getElementsByClassName())
5- por Seletor (ele nome das id's e class so que com a formataçao do css, 
quando é para declarar um id (tag#nameid) 
ou class (tag.nameclass).

querySelector(tag#nameid))

exemplo:
var divMensagem = document.querySelector('div#mensagem)
divMensagem.style.background = 'blue'

// Detalhe - toda vez que element estiver no plural, (elements) significa que ele pode pegar varias variaveis de uma vez por isto usamos : [0]
para indicar qual elemento daquele conjunto de elementos quero pegar, começando pelo numero zero.
--------------------------------------

neste tipo de situ é necessario especificar o inner:

document.write(segundoParagrafo.innerHTML)

// detalhe - InnerText só pego o texto sem os filhos 
// innerHTML pega tudo que esta envolvendo aquela tag até os filhos.
// para pegar a tag body e head náo precisa do comenado - getElement...
// primeiro declaramos uma variavel e depois mudamos a cor dela atraves da arvore STYLE.