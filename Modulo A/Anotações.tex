Regras dos indentificadores : 

1ª Podem começar com : Letras, $ ou _
2ª Não podem começar com:  Numeros
3ª pode se usar letras juntos com numeros
4ª é possivel usar acentos e simnbolos
5ª não pode conter espaços
6ª não pode usar palavras reservadas do javascript
7ª Maisculas e minusculas fazem diferença
8ª Manter nomeclatura correta

--------------------------------------------
Comandos do Terminal:
Entrar no node digite: node
para sair do node digite: .exit
para limpar o terminal: ctrl+l

--------------------------
Tipos de dados:
string
boolean
null = object
undefined
object{
    array
}
function
Number{
infinity
NaN
}
----------------------------
Comandos de tratamento de numeros:
detalhe: n é visto como uma variavel nos exemplos abaixo

Number.parseInt(n) // parseInt(n)
Number.parseFloat(n) // parseFloat(n)
Number(n)
exemplo: 
var n1 = 1545.5
n1.toFixed(2)
n1.toFixed(2).replace('.', ',')

// localizar a string e tranforma-la em moeda do local...
n1.toLocaleString('pt-BR',{style:'currency', currency:'BRL'})
n1.toLocaleString('pt-BR',{style:'currency', currency:'USD'})
n1.toLocaleString('pt-BR',{style:'currency', currency:'EUR'})

Comandos de tratamento de string:
String (n)
n.toString()
n.lenght // Quantas caracteres uma string em uma varivavel ou array
n.toUpperCase() // tranformar uma string totalmente em letras maiusculas
n.toLowerCase() // tranformar uma string totalmente em letras minusculas

tratamento de formatação de strings + variaveis:
alert ("eu sou"+n+" prazer em conhece-la") // uso de contentanação
alert (`Eu sou ${n}$, prazer em conhece-la`) // template string

---------
Comandos de pesquisa:

typeof == é para saber que tipo de dado ou variavel é

----------------------------
Comentarios no: 
HTML:
<!-- -->

JS:
//
/* */

CSS:
/* */