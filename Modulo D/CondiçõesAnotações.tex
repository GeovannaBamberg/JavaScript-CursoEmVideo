Montagem de condição composta
if (condição) {

} else{

}

Montagem de condição composta elseif:

if (condição) {

} elseif (condição){

}else{

}

Condição aninhada:

if (condition) {
            
    } else {
        if() {

        }else{
            if(as) {

            }
        }

    }

condições multiplas sem ser if:

switch (expressão) {
    case 1:
        
        break;
    case value:
        break

    default:
        break;
}
detalhe: no switch é obrigatorio ter dois pontos após relatar o value e abrir o bloco daquele caso,
e obrigatorio ter um break no fim de cada bloco case.
obs: default funciona como else, que não tem value exato, ele acontece quando não funciona nenhum dos cases.

---------- Objeto incluso -----------
switch é muito importante para usar em dados pontuais, if e else mais para intervalos 
switch só funciona com numeros inteiros e strings.

exemplo{
    var hoje = new Date() 
    var diaDaSemana = hoje.getDay()
    console.log (diaDaSemana)

    resposta do console log : 5
}
new Date (): é um objeto incluso do js, ele tem incluindo pegar horarios em tempo real (do user), data, dias da semana, ano... etc.

mas o comando objeto precisa sempre, primeiro ser declado (chamado), e depois ser especificado 
como o exemplo a cima, a especificação seria diaDaSemana.

o comando : hoje.getDay(), ele pega o dia da semana (segunda, terça.. etc), só que em valor numerico e não string, então é (domingo = 0,segunda = 1, terça=2... sexta =5) e assim vai.

o comando: hoje.getDate(), pega o dia da data atual.
    hoje.getHours(), pega a hora e assim vai.
